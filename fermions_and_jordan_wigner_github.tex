\documentclass[12pt]{article}

%#texweb_version 1.0
\newcommand{\post}{section }

\begin{document}

\title{The Fermionic canonical commutation relations and the
  Jordan-Wigner transform} 

\author{Michael A. Nielsen\thanks{mn@michaelnielsen.org and
    http://michaelnielsen.org} }

\date{\today}

\maketitle

\emph{These notes introduce Fermi algebras and a powerful tool known
  as the Jordan-Wigner transform, which allows one to move back and
  forth between describing a system as a collection of qubits, and as
  a collection of fermions.  The notes assume familiarity with
  elementary quantum mechanics, comfort with elementary linear algebra
  (but not advanced techniques), and a little familiarity with the
  basic nomenclature of quantum information science (qubits, the Pauli
  matrices).}

\emph{The notes are based on a series of blog posts I wrote, available
  online at
  http://michaelnielsen.org/blog/complete-notes-on-fermions-and-the-jordan-wigner-transform/.
  I'm now releasing the notes under a Creative Commons Attribution
  license (CC BY).  That means anyone can copy,
  distribute, transmit and adapt/remix the work, provided my
  contribution is attributed.

%#general_header: <strong>Note:</strong> This post is one in a
%# series describing fermi algebras, and a powerful tool known as
%# the <em>Jordan-Wigner transform</em>, which allows one to move back
%# and forth between describing a system as a collection of qubits, and as
%# a collection of fermions.  The posts assume familiarity
%# with elementary quantum mechanics, comfort with
%# elementary linear algebra (but not advanced techniques), and a little
%# familiarity with the basic nomenclature of quantum information science
%# (qubits, the Pauli matrices).
%#end_general_header

%#general_header: <strong>Note:</strong> This post is one in a
%# series describing fermi algebras, and a powerful tool known as
%# the <em>Jordan-Wigner transform</em>, which allows one to move back
%# and forth between describing a system as a collection of qubits, and as
%# a collection of fermions.  The posts assume familiarity
%# with elementary quantum mechanics, comfort with
%# elementary linear algebra (but not advanced techniques), and a little
%# familiarity with the basic nomenclature of quantum information science
%# (qubits, the Pauli matrices).
%#end_general_header

%#summary_post: The full 
%# <a href="http://www.qinfo.org/people/nielsen/blog/archive/">pdf
%# text</a> of my series of posts about fermi algebras and the 
%# Jordan-Wigner transform.
%#end_summary_post

%#post_break

\section{Introduction}

When you learn undergraduate quantum mechanics, it starts out being
all about wavefunctions and Hamiltonians, finding energy eigenvalues
and eigenstates, calculating measurement probabilities, and so on.

If your physics education was anything like mine, at some point a
mysterious jump occurs.  People teaching more advanced subjects, like
quantum field theory, condensed matter physics, or quantum optics,
start ``imposing canonical commutation relations'' on various field
operators.

Any student quickly realizes that ``imposing canonical commutation
relations'' is extremely important, but, speaking personally, at the
time I found it quite mysterious exactly what people meant by this
phrase.  It's only in the past few years that I've obtained a
satisfactory understanding of how this works, and understood why I had
such trouble in the first place.

These notes contain two parts.  The first part is a short tutorial
explaining the Fermionic canonical commutation relations (CCRs) from
an elementary point of view: the different meanings they can have,
both mathematical and physical, and what mathematical consequences
they have.  I concentrate more on the mathematical consequences than
the physical in these notes, since having a good grasp of the former
seems to make it relatively easy to appreciate the latter, but not so
much vice versa.  I may come back to the physical aspect in some later
notes.

The second part of the notes describes a beautiful application of the
Fermionic CCRs known as the \emph{Jordan-Wigner transform}.  This
powerful tool allows us to map a system of interacting qubits onto an
\emph{equivalent} system of interacting Fermions, or, vice versa, to
map a system of Fermions onto a system of qubits.

Why is this kind of mapping interesting?  It's interesting because it
means that anything we understand about one type of system (e.g.,
Fermions) can be immediately applied to learn something about the
other type of system (e.g., qubits).

I'll describe an application of this idea, taking what appears to be a
very complicated one-dimensional model of interacting spin-$\frac 12$
particles, and showing that it is equivalent to a simple model of
non-interacting Fermions.  This enables us to solve for the energy
spectrum and eigenstates of the original Hamiltonian.  This has, of
course, intrinsic importance, since we'd like to understand such spin
models --- they're important for a whole bundle of reasons, not the
least of which is that they're perhaps the simplest systems in which
quantum phase transitions occur.  But this example is only the tip of
a much larger iceberg: the idea that the best way of understanding
some physical systems may be to map those systems onto mathematically
equivalent but physically quite different systems, whose properties we
already understand.  Physically, we say that we introduce a
\emph{quasiparticle description} of the original system, in order to
simplify its understanding.  This idea has been of critical importance
in much of modern physics, including the understanding of
superconductivity and the quantum Hall effect.

Another application of the Jordan-Wigner transform, which I won't
describe in detail here, but which might be of interest to quantum
computing people, is to the quantum simulation of a system of
Fermions.  In particular, the Jordan-Wigner transform allows us to
take a system of interacting Fermions, and map it onto an equivalent
model of interacting spins, which can then, in principle, be simulated
using standard techniques on a quantum computer.  This enables us to
use quantum computers to efficiently simulate systems of interacting
Fermions.  This is not a trivial problem, as can be seen from the
following quote from Feynman, in his famous 1982 paper on quantum
computing:
\begin{quote}
  ``[with Feynman's proposed quantum computing device] could we
  imitate every quantum mechanical system which is discrete and has a
  finite number of degrees of freedom?  I know, almost certainly, that
  we could do that for any quantum mechanical system which involves
  Bose particles.  I'm not sure whether Fermi particles could be
  described by such a system.  So I leave that open.''
\end{quote}
It wasn't until 20 years later, and the work by Somma, Ortiz,
Gubernatis, Knill and Laflamme (Physical Review A, 2002) that this
problem was resolved, by making use of the Jordan-Wigner transform.

%#post_footer: The next post will introduce the canonical commutation 
%# relations for
%# fermions, and discuss something of their mathematical and physical
%# significance.
%#end_post_footer

%#post_break

%#post_header: <strong>Todays's post</strong> introduces the canonical
%#  commutation relations
%# fermions, and discusses something of their mathematical and physical
%# significance.
%#end_post_header

\section{Fermions}

\subsection{The canonical commutation relations for Fermions}

Suppose we have a set of operators $a_1,\ldots,a_n$ acting on some
Hilbert space $V$.  Then we say that these operations satisfy the
\emph{canonical commutation relations (CCRs) for Fermions} if they
satisfy the equations
\begin{eqnarray}
  \{ a_j, a_k^\dagger \} = \delta_{jk} I; \,\,\,\, \{ a_j,a_k \} = 0,
\end{eqnarray}
where $\{ A, B \} \equiv AB+BA$ is the anticommutator.  Note that when
we take the conjugate of the second of these relations we obtain $\{
a_j^\dagger, a_k^\dagger \} = 0$, which is sometimes also referred to
as one of the CCRs.  It is also frequently useful to set $j = k$,
giving $a_j^2 = (a_j^\dagger)^2 = 0$.

How should one understand the CCRs?  One way of thinking about the
CCRS is in an axiomatic mathematical sense.  In this way of thinking
they are purely mathematical conditions that can be imposed on a set
of matrices: for a given set of matrices, we can simply check and
verify whether those matrices satisfy or do not satisfy the CCRs.  For
example, when the state space $V$ is that of a single qubit, we can
easily verify that the operator $a = |0\rangle \langle 1|$ satisfies
the Fermionic CCRs.  From this axiomatic point of view the question to
ask is what consequences about the structure of $V$ and the operators
$a_j$ can be deduced from the fact that the CCRs hold.  

There's also a more sophisticated (but still entirely mathematical)
way of understanding the CCRs, as an instance of the relationship
between abstract algebraic objects (such as groups, Lie algebras, or
Hopf algebras), and their representations as linear maps on vector
spaces.  My own knowledge of representation theory is limited to a
little representation theory of finite groups and of Lie algberas, and
I certainly do not see the full context in the way an expert on
representation theory would.  However, even with that limited
background, one can see that there are common themes and techniques:
what may appear to be an isolated technique or trick is often really
an instance of a much deeper idea or set of ideas that become obvious
once once one has enough broad familiarity with representation theory.
I'm not going to pursue this point of view in these notes, but thought
it worth mentioning for the sake of giving context and motivation to
the study of other topics.

Finally, there's a physical way in which we can understand the CCRs.
When we want to describe a system containing Fermions, one way to
begin is to start by writing down a set of operators satisfying the
CCRs, and then to try to guess what sort of Hamiltonian involving
those operators could describe the interactions observed in the
system, often motivated by classical considerations, or other rules of
thumb. This is, for example, the sort of point of view pursued in the
BCS theory of superconductivity, and which people are trying to pursue
in understanding high temperature superconductors.

Of course, one can ask why physicists want to use operators satisfying
the Fermionic CCRs to describe a system of Fermions, or why anyone,
mathematician or physicist, would ever write down the CCRs in the
first place.  These are good questions, which I'm not going to try to
answer here, although one or both questions might make a good subject
for some future notes.  (It is, of course, a \emph{lot} easier to
answer these questions once you understand the material I present
here.)

Instead, I'm going to approach the Fermionic CCRs from a purely
mathematical point of view, asking the question ``What can we deduce
from the fact that a set of operators satisfying the CCRs exists?''
The surprising answer is that we can deduce quite a lot about the
structure of $V$ and the operators $a_j$ simply from the fact that the
$a_j$ satisfy the canonical commutation relations!

%#post_footer: We'll take this subject up in the next post, where we'll
%# show that the CCRs essentially uniquely determine the action of
%# the [tex]a_j[/tex]s, up to a choice of basis.
%#end_post_footer

%#post_break

%#post_header: <strong>In today's post</strong> we'll
%# show that the CCRs essentially uniquely determine the action of
%# the fermionic operators [tex]a_j[/tex], up to a choice of basis.  Mathematically, the
%# argument is somewhat detailed, but it's also the kind of argument
%# that rewards detailed study, especially if studied in conjunction
%# with related topics, such as the representation theory of
%# [tex]su(2)[/tex].  You'll need to look elsewhere for that, however!
%#end_post_header

\subsection{Consequences of the fermionic CCRs}

We will assume that the vector space $V$ is finite dimensional, and
that there are $n$ operators $a_1,\ldots,a_n$ acting on $V$ and
satisfying the Fermionic CCRs.  At the end of this paragraph we're
going to give a broad outline of the steps we go through.  Upon a
first read, some of these steps may appear a little mysterious to the
reader not familiar with representation theory.  In particular, please
don't worry if you get a little stuck in your understanding of the
outline at some points, as the exposition is very much at the
bird's-eye level, and not all detail is visible at that level.
Nonetheless, the reason for including this broad outline is the belief
that repeated study will pay substantial dividends, if it is read in
conjunction with the more detailed exposition to follow, or similar
material on, e.g., representations of the Lie algebra $su(2)$.
Indeed, the advantage of operating at the bird's-eye level is that it
makes it easier to see the connections between these ideas, and the
use of similar ideas in other branches of representation theory.
\begin{itemize}
\item We'll start by showing that the operators $a_j^\dagger a_j$ are
  positive Hermitian operators with eigenvalues $0$ and $1$.
  
\item We'll show that $a_j$ acts as a \emph{lowering operator} for
  $a_j^\dagger a_j$, in the sense that if $|\psi\rangle$ is a
  normalized eigenstate of $a_j^\dagger a_j$ with eigenvalue $1$, then
  $a_j |\psi\rangle$ is a normalized eigenstate of $a_j^\dagger a_j$
  with eigenvalue $0$.  If $|\psi\rangle$ is a normalized eigenstate
  of $a_j^\dagger a_j$ with eigenvalue $0$, then $a_j |\psi\rangle$
  vanishes.
  
\item Similarly, $a_j^\dagger$ acts as a \emph{raising operator} for
  $a_j^\dagger a_j$, in the sense that if $|\psi\rangle$ is a
  normalized eigenstate of $a_j^\dagger a_j$ with eigenvalue $0$, then
  $a_j^\dagger |\psi\rangle$ is a normalized eigenstate of
  $a_j^\dagger a_j$ with eigenvalue $1$.  If $|\psi\rangle$ is a
  normalized eigenstate of $a_j^\dagger a_j$ with eigenvalue $1$, then
  $a_j^\dagger |\psi\rangle$ vanishes.

  
\item We prove that the operators $a_j^\dagger a_j$ form a mutually
  commuting set of Hermitian matrices, and thus there exists a state
  $|\psi\rangle$ which is a simultaneous eigenstate of $a_j^\dagger
  a_j$ for \emph{all} values $j=1,\ldots,n$.
  
\item By raising and lowering the state $|\psi\rangle$ in all possible
  combinations, we'll construct a set of $2^n$ orthonormal states
  which are simultaneous eigenstates of the $a_j^\dagger a_j$.  The
  corresponding vector of eigenvalues uniquely labels each state in
  this orthonormal basis.
  
\item Suppose the vector space spanned by these $2^n$ simultaneous
  eigenstates is $W$.  At this point, we know that $a_j$ and
  $a_j^\dagger$ map $W$ into $W$, and, indeed, we know everything
  about the action $a_j$ and $a_j^\dagger$ have on $W$.
  
\item Suppose we define $W_\perp$ to be the orthocomplement of $W$ in
  $V$.  Then we'll show that the $a_j$ and $a_j^\dagger$ map $W_\perp$
  into itself, and their restrictions to $W_\perp$ satisfy the
  Fermionic CCRs.  We can then repeat the above procedure, and
  identify a $2^n$-dimensional subspace of $W_\perp$ on which we know
  the action of the $a_j$ and $a_j^\dagger$ exactly.
  
\item We iterate this procedure until $W_\perp$ is the trivial vector
  space, at which point it is no longer possible to continue.  At this
  point we have established an orthonormal basis for the whole of $V$
  with respect to which we can explicilty write down the action of both
  $a_j$ and $a_j^\dagger$.
\end{itemize}

Let's go through each of these steps in more detail.

\textbf{The $a_j^\dagger a_j$ are positive Hermitian with eigenvalues
  $0$ and $1$:} Observe that the $a_j^\dagger a_j$ are positive (and
thus Hermitian) matrices.  We will show that $(a_j^\dagger a_j)^2 =
a_j^\dagger a_j$, and thus the eigenvalues of $a_j^\dagger a_j$ are
all $0$ or $1$.

To see this, observe that $(a_j^\dagger a_j)^2 = a_j^\dagger a_j
a_j^\dagger a_j = -(a_j^\dagger)^2 a_j^2 + a_j^\dagger a_j$, where we
used the CCR $\{a_j,a_j^\dagger \} = I$.  Note also that $a_j^2 = 0$
by the CCR $\{a_j,a_j\} = 0$.  It follows that $(a_j^\dagger a_j)^2 =
a_j^\dagger a_j$, as claimed.

\textbf{The $a_j$ are lowering operators:} Suppose $|\psi\rangle$ is a
normalized eigenstate of $a_j^\dagger a_j$ with eigenvalue $1$.  Then
we claim that $a_j|\psi\rangle$ is a normalized eigenstate of
$a_j^\dagger a_j$ with eigenvalue $0$.  To see that $a_j|\psi\rangle$
is normalized, note that $\langle \psi|a_j^\dagger a_j |\psi\rangle =
\langle \psi|\psi\rangle = 1$, where we used the fact that
$|\psi\rangle$ is an eigenstate of $a_j^\dagger a_j$ with eigenvalue
$1$ to establish the first equality.  To see that it has eigenvalue
$0$, note that $a_j^\dagger a_j a_j|\psi\rangle = 0$, since $\{
a_j,a_j \} = 0$.

\textbf{Exercise:} Suppose $|\psi\rangle$ is a normalized eigenstate
of $a_j^\dagger a_j$ with eigenvalue $0$.  Show that $a_j |\psi\rangle
= 0$.

\textbf{The $a_j$ are raising operators:} Suppose $|\psi\rangle$ is a
normalized eigenstate of $a_j^\dagger a_j$ with eigenvalue $0$.  Then
we claimed that $a_j^\dagger|\psi\rangle$ is a normalized eigenstate
of $a_j^\dagger a_j$ with eigenvalue $1$.

To see the normalization, we use the CCR $\{ a_j,a_j^\dagger \} = I$
to deduce $\langle \psi|a_j a_j^\dagger |\psi\rangle = -\langle
\psi|a_j^\dagger a_j|\psi\rangle + \langle \psi|\psi\rangle$.  But
$a_j^\dagger a_j|\psi\rangle = 0$, by the eigenvalue assumption, and
$\langle \psi|\psi\rangle = 1$, whence $\langle \psi|a_j a_j^\dagger
|\psi\rangle = 1$, which is the desired normalization condition.

To see that $a_j^\dagger |\psi\rangle$ is an eigenstate with
eigenvalue $1$, use the CCR $\{a_j,a_j^\dagger \} = I$ to deduce that
$a_j^\dagger a_j a_j^\dagger |\psi\rangle = - a_j^\dagger a_j^\dagger
a_j|\psi\rangle + a_j^\dagger|\psi\rangle = a_j^\dagger |\psi\rangle$,
where the final equality can be deduced either from the assumption
that $a_j^\dagger a_j |\psi\rangle = 0$, or from the CCR $\{
a_j^\dagger,a_j^\dagger \} = 0$.  This is the desired eigenvalue
equation for $a_j^\dagger|\psi\rangle$.

\textbf{Exercise:} Suppose $|\psi\rangle$ is a normalized eigenstate
of $a_j^\dagger a_j$ with eigenvalue $1$.  Show that $a_j^\dagger
|\psi\rangle = 0$.

\textbf{The $a_j^\dagger a_j$ form a mutually commuting set of
  observables:} To see this, let $j \neq k$, and apply the CCRs
repeatedly to obtain $a_j^\dagger a_j a_k^\dagger a_k = a_k^\dagger
a_k a_j^\dagger a_j$, which is the desired commutativity.

\textbf{Existence of a common eigenstate:} It is well known that a
mutually commuting set of Hermitian operators possesses a common
eigenbasis.  This fact is usually taught in undergraduate quantum
mechanics courses; for completeness, I've included a simple proof in
an appendix to these notes.  We won't make use of the full power of
this result here, but instead simply use the fact that there exists a
normalized state $|\psi\rangle$ which is a simultaneous eigenstate of
all the $a_j^\dagger a_j$ operators.  In particular, for all $j$ we
have:
\begin{eqnarray}
  a_j^\dagger a_j |\psi\rangle = \alpha_j |\psi\rangle,
\end{eqnarray}
where for each $j$ either $\alpha_j = 0$ or $\alpha_j = 1$.  It will
be convenient to assume that $\alpha_j = 0$ for all $j$.  This
assumption can be made without loss of generality, by applying
lowering operators to the $|\psi\rangle$ for each $j$ such that
$\alpha_j = 1$, resulting in a normalized state $|\mbox{vac}\rangle$
such that $a_j^\dagger a_j |\mbox{vac}\rangle = 0$ for all $j$.

\textbf{Defining an orthonormal basis:} For any vector $\alpha =
(\alpha_1,\ldots,\alpha_n)$, where each $\alpha_j = 0$ or $1$, define
a corresponding state:
\begin{eqnarray}
  |\alpha \rangle \equiv (a_1^\dagger)^{\alpha_1} \ldots (a_n^\dagger)^{\alpha_n}|\mbox{vac}\rangle.
\end{eqnarray}
It is clear that there are $2^n$ such states $|\alpha\rangle$, and
that they form an orthonormal set spanning a subspace of $V$ that we
shall call $W$.

\textbf{The action of the $a_j$ and $a_j^\dagger$ on $W$:} How do
$a_j$ and $a_j^\dagger$ act on $W$?  Stated another way, how do they
act on the orthonormal basis we have constructed for $W$, the states
$|\alpha\rangle$?  Applying the CCRs and the definition of the states
$|\alpha\rangle$ it is easy to verify that the action of $a_j$ is as
follows:
\begin{itemize}
\item Suppose $\alpha_j = 0$.  Then $a_j|\alpha\rangle = 0$.
  
\item Suppose $\alpha_j = 1$.  Let $\alpha'$ be that vector which
  results when the $j$th entry of $\alpha$ is changed to $0$.  Then
  $a_j|\alpha\rangle = -(-1)^{s_\alpha^j} |\alpha'\rangle$, where
  $s_\alpha^j \equiv \sum_{k=1}^{j-1} \alpha_k$.
\end{itemize}

The action of $a_j^\dagger$ on $W$ is similar:
\begin{itemize}
\item Suppose $\alpha_j = 0$.  Let $\alpha'$ be that vector which
  results when the $j$th entry of $\alpha$ is changed to $1$.  Then
  $a_j^\dagger|\alpha\rangle = -(-1)^{s_\alpha^j}|\alpha'\rangle$,
  where $s_\alpha^j \equiv \sum_{k=1}^{j-1} \alpha_k$.

\item Suppose $\alpha_j = 1$.  Then $a_j^\dagger |\alpha\rangle = 0$.
\end{itemize}

\textbf{Action of $a_j$ and $a_j^\dagger$ on $W_\perp$:} We have
described the action of the $a_j$ and the $a_j^\dagger$ on the
subspace $W$.  What of the action of these operators on the remainder
of $V$?  To answer that question, we first show that $a_j$ and
$a_j^\dagger$ map the orthocomplement $W_\perp$ into itself.

To see this, let $|\psi\rangle \in W_\perp$, and consider
$a_j|\psi\rangle$.  We wish to show that $a_j|\psi\rangle \in W_\perp$
also, i.e., that for any $|\phi\rangle \in W$ we have $\langle
\phi|a_j|\psi\rangle = 0$.  This follows easily by considering the
complex conjugate quantity $\langle \psi|a_j^\dagger |\phi\rangle$,
and observing that $a_j^\dagger |\phi\rangle \in W$, since
$|\phi\rangle \in W$, and thus $\langle \psi|a_j^\dagger |\phi\rangle
= 0$.  A similar argument shows that $a_j^\dagger$ maps $W_\perp$ into
itself.

Consider now the operators $\tilde a_j$ obtained by restricting $a_j$
to $W_\perp$.  Provided $W_\perp$ is nontrivial it is clear that these
operators satisfy the CCRs on $W_\perp$.  Repeating the above
argument, we can therefore identify a $2^n$-dimensional subspace of
$W_\perp$ on which we can compute the action of $\tilde a_j$ and
$\tilde a_j^\dagger$, and thus of $a_j$ and $a_j^\dagger$.

We may iterate this procedure many times, but the fact that $V$ is
finite dimensional means that the process must eventually terminate.
At the point of termination we will have broken up $V$ as a direct sum
of some finite number $d$ of orthonormal $2^n$-dimensional vector
spaces, $W_1,W_2,\ldots,W_d$, and on each vector space we will have an
orthonormal basis with respect to which the action of $a_j$ and
$a_j^\dagger$ is known precisely.

Stated another way, we can introduce an orthonormal basis
$|\alpha,k\rangle$ for $V$, where $\alpha$ runs over all $n$-bit
vectors, and $k = 1,\ldots,d$, and such that the action of the $a_j$
and $a_j^\dagger$ is to leave $k$ invariant, and to act on
$|\alpha\rangle$ as described above.  In this representation it is
clear that $V$ can be regarded as a tensor product $C^{2^n} \otimes
C^d$, with the action of $a_j$ and $a_j^\dagger$ trivial on the $C^d$
component.  We will call this the \emph{occupation number
  representation} for the Fermi algebra $a_j$.

It's worth pausing to appreciate what has been achieved here: starting
only from the CCRs for $a_1,\ldots,a_n$ we have proved that $V$ can be
broken down into a tensor product of a $2^n$-dimensional vector space
and a $d$-dimensional vector space, with the $a_j$s acting
nontrivially only on the $2^n$-dimensional component. Furthermore, the
action of the $a_j$s is completely known.  I think it's quite
remarkable that we can say so much: at the outset it wasn't even
obvious that the dimension of $V$ should be a multiple of $2^n$!

When $d=1$ we will call this the \emph{fundamental representation} for
the Fermionic CCRs.  (This is the terminology I use, but I don't know
if it is standard or not.)  Up to a change of basis it is clear that
all other representations can be obtained by taking a tensor product
of the fundamental representation with the trivial action on a
$d$-dimensional vector space.

%#post_footer: We've now understood the fundamental mathematical fact about 
%# fermions: the mere existence of operators satisfying Fermionic
%# canonical commutation relations <em>completely determines</em> the action of
%# those operators with respect to some suitable
%# orthonormal occuptation number basis.  That's a very strong statement!
%# In the next post we'll use this technology to study a problem of
%# direct physical interest: finding the energy spectrum and eigenstates
%# of a Hamiltonian which is quadratic in terms of Fermi operators.
%#end_post_footer

%#post_break

%#post_header: <strong>In today's post</strong> we'll learn how to
%# diagonalize a Hamiltonian which is quadratic in operators satisfying
%# the Fermionic CCRs.  Remarkably, we'll do this using only the CCRs:
%# the operators could arise in many different ways physically, but,
%# as we shall see, it is only the CCRs that matter for determining the
%# spectrum!
%#end_post_header

\subsection{Diagonalizing a Fermi quadratic Hamiltonian}

Suppose $a_1,\ldots,a_n$ satisfy the Fermionic CCRs, and we have a
system with Hamiltonian
\begin{eqnarray}
  H_{\rm free} = \sum_j \alpha_j a_j^\dagger a_j,
\end{eqnarray}
where $\alpha_j \geq 0$ for each value of $j$.  In physical terms,
this is the Hamiltonian used to describe a system of free, i.e.,
non-interacting, Fermions.  

Such Hamiltonians are used, for example, in the simplest possible
quantum mechanical model of a metal, the Drude-Sommerfeld model, which
treats the conduction electrons as free Fermions.  Such a model may
appear pretty simplistic (especially after we solve it, below), but
actually there's an amazing amount of physics one can get out of such
simple models.  I won't dwell on these physical consequences here, but
if you're unfamiliar with the Drude-Sommerfeld theory, you could
profitably spend a couple of hours looking at the first couple of
chapters in a good book on condensed matter physics, like Ashcroft and
Mermin's ``Solid State Physics'', which explains the Drude-Sommerfeld
model and its consequences in detail.  (Why such a simplistic model
does such a great job of describing metals is another long story,
which I may come back to in a future post.)

Returning to the abstract Hamiltonian $H_{\rm free}$, the positivity
of the operators $a_j^\dagger a_j$ implies that $\langle \psi |H_{\rm
  free} |\psi\rangle \geq 0$ for any state $|\psi\rangle$, and thus
the ground state energy of $H_{\rm free}$ is non-negative.  However,
our earlier construction also shows that we can find at least one
state $|\mbox{vac}\rangle$ such that $a_j^\dagger
a_j|\mbox{vac}\rangle = 0$ for all $j$, and thus $H_{\rm
  free}|\mbox{vac}\rangle = 0$.  It follows that the ground state
energy of $H_{\rm free}$ is exactly $0$.

This result is easily generalized to the case where the $\alpha_j$
have any sign, with the result that the ground state energy is $\sum_j
\min(0,\alpha_j)$, and the ground state $|\psi\rangle$ is obtained
from $|\mbox{vac}\rangle$ by applying the raising operator $a_j$ for
all $j$ with $\alpha_j < 0$. More generally, the allowed energies of
the excited states of this system correspond to sums over subsets of
the $\alpha_j$.

\textbf{Exercise:} Express the excited states of the system in terms
of $|\mbox{vac}\rangle$.

Just by the way, readers with an interest in computational complexity
theory may find it interesting to note a connection between the
spectrum of $H_{\rm free}$ and the \emph{Subset-Sum} problem from
computer science.  The \emph{Subset-Sum} problem is this: given a set
of integers $x_1,\ldots,x_n$, with repetition allowed, is there a
subset of those integers which adds up to a desired target, $t$?
Obviously, the problem of determining whether $H_{\rm free}$ has a
particular energy is equivalent to the \emph{Subset-Sum} problem, at
least in the case where the $\alpha_j$ are integers.  What is
interesting is that the \emph{Subset-Sum} problem is known to be
\textbf{NP-Complete}, in the language of computational complexity
theory, and thus is regarded as computationally intractable.  As a
consequence, we deduce that the problem of determining whether a
particular value for energy is in the spectrum of $H_{\rm free}$ is in
general \textbf{NP-Hard}, i.e., at least as difficult as the
\textbf{NP-Complete} problems.  Similar results hold for the more
general Fermi Hamiltonians considered below.  Furthermore, this
observation suggests the possibility of an interesting link between
the physical problem of estimating the density of states, and classes
of problems in computational complexity theory, such as the counting
classes (e.g., \textbf{\#P}), and also to approximation problems.

Let's generalize our results about the spectrum of $H_{\rm free}$.
Suppose now that we have the Hamiltonian
\begin{eqnarray}
  H = \sum_{jk} \alpha_{jk} a_j^\dagger a_k. 
\end{eqnarray}
Taking the adjoint of this equation we see that in order for $H$ to be
hermitian, we must have $\alpha_{jk}^* = \alpha_{kj}$, i.e., the
matrix $\alpha$ whose entries are the $\alpha_{jk}$ is itself
hermitian.

Suppose we introduce new operators $b_1,\ldots,b_n$ defined by
\begin{eqnarray}
  b_j \equiv \sum_{k=1}^n \beta_{jk} a_k,
\end{eqnarray}
where $\beta_{jk}$ are complex numbers.  We are going to try to choose
the $\beta_{jk}$ so that (1) the operators $b_j$ satisfy the Fermionic
CCRs, and (2) when expressed in terms of the $b_j$, the Hamiltonian
$H$ takes on the same form as $H_{\rm free}$, and thus can be
diagonalized.

We begin by looking for conditions on the complex numbers $\beta_{jk}$
such that the $b_j$ operators satisfy Fermionic CCRs.  Computing
anticommutators we find
\begin{eqnarray}
  \{ b_j, b_k^\dagger \} = \sum_{lm} \beta_{jl} \beta_{km}^* \{ a_l,a_m^\dagger \}.
\end{eqnarray}
Substituting the CCR $\{ a_l,a_m^\dagger \} = \delta_{lm} I$ and
writing $\beta_{km}^* = \beta_{mk}^\dagger$ gives
\begin{eqnarray}
  \{ b_j, b_k^\dagger \} = \sum_{lm} \beta_{jl} \delta_{lm} \beta_{mk}^\dagger I= (\beta \beta^\dagger)_{jk} I
\end{eqnarray}
where $\beta \beta^\dagger$ denotes the matrix product of the matrix
$\beta$ with entries $\beta_{jl}$ and its adjoint $\beta^\dagger$. To
compute $\{b_j,b_k\}$ we use the linearity of the anticommutator
bracket in each term to express $\{b_j,b_k\}$ as a sum over terms of
the form $\{ a_l,a_m \}$, each of which is $0$, by the CCRs.  As a
result, we have:
\begin{eqnarray}
  \{ b_j,b_k \} = 0.
\end{eqnarray}
It follows that provided $\beta \beta^\dagger = I$, i.e., provided
$\beta$ is unitary, the operators $b_j$ satisfy the Fermionic CCRs.

Let's assume that $\beta$ is unitary, and change our notation, writing
$u_{jk} \equiv \beta_{jk}$ in order to emphasize the unitarity of this
matrix.  We now have
\begin{eqnarray}
  b_j = \sum_k u_{jk} a_k.
\end{eqnarray}
Using the unitarity of $u$ we can invert this equation to obtain
\begin{eqnarray}
  a_j = \sum_k u^\dagger_{jk} b_k.
\end{eqnarray}
Substituting this expression and its adjoint into $H$ and doing some
simplification gives us
\begin{eqnarray}
  H = \sum_{lm} (u \alpha u^\dagger)_{lm} b_l^\dagger b_m.
\end{eqnarray}
Since $\alpha$ is hermitian, we can choose $u$ so that $u \alpha
u^\dagger$ is diagonal, with entries $\lambda_j$, the eigenvalues of
$\alpha$, giving us
\begin{eqnarray}
  H = \sum_j \lambda_j b_j^\dagger b_j.
\end{eqnarray}
This is of the same form as $H_{\rm free}$, and thus the ground state
energy and excitation energies may be computed in the same way as we
described earlier.

What about the ground state of $H$?  Assuming that all the $\lambda_j$
are non-negative, it turns out that a state $|\psi\rangle$ satisfies
$a_j^\dagger a_j |\psi\rangle = 0$ for all $j$ if and only if
$b_j^\dagger b_j|\psi\rangle = 0$ for all $j$, and so the ground state
for the two sets of Fermi operators is the same.


This follows from a more general observation, namely, that
$a_j^\dagger a_j |\psi\rangle = 0$ if and only if $a_j|\psi\rangle =
0$.  In one direction, this is trivial: just multiply $a_j|\psi\rangle
= 0$ on the left by $a_j^\dagger$.  In the other direction, we
multiply $a_j^\dagger a_j |\psi\rangle = 0$ on the left by $a_j$ to
obtain $a_j a_j^\dagger a_j |\psi\rangle = 0$.  Substituting the CCR
$a_j a_j^\dagger = -a_j^\dagger a_j + I$, we obtain
\begin{eqnarray}
  (-a_j^\dagger a_j^2+a_j)|\psi\rangle = 0.
\end{eqnarray}
But $a_j^2 = 0$, so this simplifies to $a_j|\psi\rangle = 0$, as
desired.

Returning to the question of determining the ground state, supposing
$a_j^\dagger a_j|\psi\rangle = 0$ for all $j$, we immediately have
$a_j|\psi\rangle = 0$ for all $j$, and thus $b_j|\psi\rangle = 0$ for
all $j$, since the $b_j$ are linear functions of the $a_j$, and thus
$b_j^\dagger b_j|\psi\rangle = 0$ for all $j$.  This shows that the
ground state for the two sets of Fermi operators, $a_j$ and $b_j$, is
in fact the same.  The excitations for $H$ may be obtained by applying
raising operators $b_j^\dagger$ to the ground state.

\textbf{Exercise:} Suppose some of the $\lambda_j$ are negative.
Express the ground state of $H$ in terms of the simultaneous
eigenstates of the $a_j^\dagger a_j$.

%#post_footer: Okay, that's enough for one day!  We've learnt how to
%# diagonalize a fairly general class of Hamiltonians quadratic in
%# Fermi operators.  In the next post we'll go further, learning
%# how to cope with additional terms like [tex]a_j a_j[/tex] and
%# [tex]a_j^\dagger a_k^\dagger[/tex].
%#end_post_footer

%#post_break

%#post_header: <strong>In today's post</strong> we continue with the
%# train of thought begun in the last post, learning how to
%# find the energy spectrum of any Hamiltonian quadratic in Fermi operators.
%# Although we don't dwell in this post on the connection to specific physical
%# models, this class of Hamiltonians
%# covers an enormous number of models of relevance in condensed matter
%# physics.  In later posts we'll apply these results (together with the
%# Jordan-Wigner transform) to understand the energy spectrum of some
%# models of interacting spins in one dimension, which are prototypes for
%# an understanding of quantum magnets and many, many other phenomena,
%# including many systems which undergo quantum phase transitions.
%#end_post_header

The Hamiltonian $H = \sum_{jk} \alpha_{jk} a_j^\dagger a_k$ we
diagonalized earlier can be generalized to any Hamiltonian which is
quadratic in Fermi operators, by which we mean it may contain terms of
the form $a^\dagger_j a_k, a_j a_k^\dagger, a_j a_k$ and $a_j^\dagger
a_k$.  We will not allow linear terms like $a_j+a_j^\dagger$.
Additive constant terms $\gamma I$ are easily incorporated, since they
simply displace all elements of the spectrum by an amount $\gamma$.
There are several ways one can write such a Hamiltonian, but the
following form turns out to be especially convenient for our purposes:
\begin{eqnarray}
  H = \sum_{jk} \left( \alpha_{jk} a_j^\dagger a_k -\alpha^*_{jk} a_j a_k^\dagger +
  \beta_{jk} a_j a_k - \beta^*_{jk} a_j^\dagger a_k^\dagger \right).
\end{eqnarray}
The reader should spend a little time convincing themselves that for
the class of Hamiltonians we have described, it is always possible to
write the Hamiltonian in this form, up to an additive constant $\gamma
I$, and with $\alpha$ hermitian and $\beta$ antisymmetric.

This class of Hamiltonian appears to have first been diagonalized in
an appendix to a famous \emph{Annals of Physics} paper by Lieb,
Schultz and Mattis, dating to 1961 (volume 16, pages 407-466), and the
procedure we follow is inspired by theirs.  We begin by defining
operators $b_1,\ldots,b_n$:
\begin{eqnarray}
  b_j \equiv \sum_k \left( \gamma_{jk} a_k + \mu_{jk} a_k^\dagger \right).
\end{eqnarray}
We will try to choose the complex numbers $\gamma_{jk}$ and $\mu_{jk}$
to ensure that: (1) the operators $b_j$ satisfy Fermionic CCRs; and
(2) when expressed in terms of the $b_j$, $H$ has the same form as
$H_{\rm free}$, and so can be diagonalized.

A calculation shows that the condition $\{ b_j, b_k^\dagger \} =
\delta_{jk} I$ is equivalent to the condition
\begin{eqnarray}
  \gamma \gamma^\dagger + \mu \mu^\dagger = I,
\end{eqnarray}
while the condition $\{ b_j, b_k \} = 0$ is equivalent to the condition
\begin{eqnarray}
  \gamma \mu^T+\mu \gamma^T = 0.
\end{eqnarray}
These are straightforward enough equations, but their meaning is
perhaps a little mysterious.  More insight into their structure is
obtained by rewriting the connection between the $a_j$s and the $b_j$s
in an equivalent form using vectors whose individual entries are not
numbers, but rather are operators such as $a_j$ and $b_j$, and using a
block matrix with blocks $\gamma, \mu, \mu^*$ and $\gamma^*$:
\begin{eqnarray}
  \left[ \begin{array}{c} b_1 \\ \vdots \\ b_n \\
    b_1^\dagger \\ \vdots \\ b_n^\dagger \end{array} \right]
  = \left[ \begin{array}{cc} \gamma & \mu \\ \mu^* & \gamma^*
      \end{array} \right] 
  \left[ \begin{array}{c} a_1 \\ \vdots \\ a_n \\
    a_1^\dagger \\ \vdots \\ a_n^\dagger \end{array} \right].
\end{eqnarray}
The conditions derived above for the $b_j$s to satisfy the CCRs are
equivalent to the condition that the transformation matrix
\begin{eqnarray}
  T \equiv \left[ \begin{array}{cc} \gamma & \mu \\ \mu^* & \gamma^*
      \end{array} \right]
\end{eqnarray}
is unitary, which is perhaps a somewhat less mysterious condition than
the earlier equations involving $\gamma$ and $\mu$.  One advantage of
this representation is that it makes it easy to find an expression for
the $a_j$ in terms of the $b_j$, simply by inverting this unitary
transformation, to obtain:
\begin{eqnarray}
  \left[ \begin{array}{c} a_1 \\ \vdots \\ a_n \\
    a_1^\dagger \\ \vdots \\ a_n^\dagger \end{array} \right]
  = T^\dagger
  \left[ \begin{array}{c} b_1 \\ \vdots \\ b_n \\
    b_1^\dagger \\ \vdots \\ b_n^\dagger \end{array} \right].
\end{eqnarray}

The next step is to rewrite the Hamiltonian in terms of the $b_j$
operators.  To do this, observe that:
\begin{eqnarray}
  H = [ a_1^\dagger \ldots a_n^\dagger a_1 \ldots a_n ]
  \left[ \begin{array}{cc} \alpha & -\beta^* \\ \beta & -\alpha^*
      \end{array} \right]
    \left[ \begin{array}{c} a_1 \\ \vdots \\ a_n \\ a_1^\dagger \\
        \vdots \\ a_n^\dagger \end{array} \right].
\end{eqnarray}
It is actually this expression for $H$ which motivated the original
special form which we chose for $H$.  The expression is convenient,
for it allows us to easily transform back and forth between $H$
expressed in terms of the $a_j$ and $H$ in terms of the $b_j$.  We
already have an expression in terms of the $b_j$ operators for the
column vector containing the $a$ and $a^\dagger$ terms.  With a little
algebra this gives rise to a corresponding expression for the row
vector containing the $a^\dagger$ and $a$ terms:
\begin{eqnarray}
  [a_1^\dagger \ldots a_n^\dagger a_1 \ldots a_n]
  = [b_1^\dagger \ldots b_n^\dagger b_1 \ldots b_n] T.
\end{eqnarray}
This allows us to rewrite the Hamiltonian as
\begin{eqnarray}
  H = 
  [b^\dagger b]
  T M T^\dagger
  \left[ \begin{array}{c} b \\
    b^\dagger \end{array} \right],
\end{eqnarray}
where we have used the shorthand $[b^\dagger b]$ to denote the vector
with entries $b_1^\dagger, \ldots, b_n^\dagger,b_1,\ldots,b_n$, and
\begin{eqnarray}
  M = \left[ \begin{array}{cc} \alpha & -\beta^* \\ \beta & -\alpha^*
      \end{array} \right].
\end{eqnarray}
Supposing we can choose $T$ such that $TMT^\dagger$ is diagonal, we
see that the Hamiltonian can be expressed in the form of $H_{\rm
  free}$, and the energy spectrum found, following our earlier methods.

Since $\alpha$ is hermitian and $\beta$ antisymmetric it follows that
$M$ also is hermitian, and so can be diagonalized for some choice of
unitary $T$.  However, the fact that the $b_j$s must satisfy the CCRs
constrains the class of $T$s available to us.  We need to show that
such a $T$ can be used to do the diagonalization.

We will give a heuristic and somewhat incomplete proof that this is
possible, before making some brief remarks about what is required for
a rigorous proof. I've omitted the rigorous proof, since the way I
understand it is uses a result from linear algebra that, while
beautiful, I don't want to explain in full detail here.

Suppose $T$ is any unitary such that
\begin{eqnarray}
  T M T^\dagger = \left[ \begin{array}{cc} d & 0 \\ 0 & -d \end{array} \right],
\end{eqnarray}
where $d$ is diagonal, and we used the special form of $M$ to deduce
that the eigenvalues are real and appear in matched pairs $\pm
\lambda$.  We'd like to show that $T$ can be chosen to be of the
desired special form.  To see that this is plausible, consider the map
$X \rightarrow S X^* S^\dagger$, where $S$ is a block matrix:
\begin{eqnarray}
  S = \left[ \begin{array}{cc} 0 & I \\ I & 0 \end{array} \right].
\end{eqnarray}
Applying this map to both sides of the earlier equation we obtain
\begin{eqnarray}
  ST^* M^* T^T S^\dagger = 
\left[ \begin{array}{cc} -d & 0 \\ 0 & d \end{array} \right] = -TMT^\dagger.
\end{eqnarray}
But $M^* = -S^\dagger M S$, and so we obtain:
\begin{eqnarray}
  -ST^* S^\dagger M S T^T S^\dagger = -TMT^\dagger.
\end{eqnarray}
It is at least plausible that we can choose $T$ such that
$ST^*S^\dagger = T$, which would imply that $T$ has the required form.
What this actually shows is, of course, somewhat weaker, namely that
$T^\dagger ST^* S^\dagger$ commutes with $M$.  

One way of obtaining a rigorous proof is to find a $T$ satisfying
\begin{eqnarray}
  T M T^\dagger = \left[ \begin{array}{cc} d & 0 \\ 0 & -d \end{array} \right],
\end{eqnarray}
and then to apply the cosine-sine (or CS) decomposition from linear
algebra, which provides a beautiful way of representing block unitary
matrices, and which, in this instance, allows us to obtain a $T$ of
the desired form with just a little more work.  The CS decomposition
may be found, for example, as Theorem VII.1.6 on page 196 of Bhatia's
book ``Matrix Analysis'' (Springer-Verlag, New York, 1997).

\textbf{Problem:} Can we extend these results to allow terms in the
Hamiltonian which are \emph{linear} in the Fermi operators?

%#post_footer: In this post we've seen how to diagonalize a general Fermi
%# quadratic Hamiltonian.  We've treated this as a purely
%# mathematical problem, although most physicists will probably have little trouble
%# believing that these techniques are useful in a wide range of physical
%# problems.  In the next post we'll explain a surprising connection between
%# these ideas and one-dimensional spin systems: a tool known as the
%# <em>Jordan-Wigner transform</em> can be used to establish an equivalence
%# between a large class of one-dimensional spin systems and the type
%# of Fermi systems we have been considering.  This is interesting
%# because included in this class of one-dimensional spin systems are 
%# important models such as the transverse Ising model, which serve
%# as a general prototype for quantum magnetism, are a good basis for
%# understanding some naturally occurring physical systems, and which
%# also serve as prototypes for the understanding of quantum phase
%# transitions.
%#end_post_footer

%#post_break

%#post_header: <strong>In today's post</strong>
%# we'll discuss a surprising connection between
%# the Fermionic ideas we've been discussing up to now, and one-dimensional 
%# spin systems.  In particular, a tool known as the
%# <em>Jordan-Wigner transform</em> can be used to establish an equivalence
%# between a large class of one-dimensional spin systems and the type
%# of Fermi systems we have been considering.  This is interesting
%# because included in this class of one-dimensional spin systems are 
%# important models such as the transverse Ising model, which serve
%# as a general prototype for quantum magnetism, are a good basis for
%# understanding some naturally occurring physical systems, and which
%# also serve as prototypes for the understanding of quantum phase
%# transitions.
%#end_post_header

\section{The Jordan-Wigner transform}

In this section we describe the Jordan-Wigner transform, explaining how
it can be used to map a system of qubits (i.e., spin-$\frac 12$
systems) to a system of Fermions, and vice versa.  We also explain a
nice applications of these ideas, to solving one-dimensional quantum
spin systems.

Suppose we have an $n$-qubit system, with the usual state space
$C^{2^n}$, and Pauli operators $X_j, Y_j,Z_j$ acting on qubit $j$.  We
are going to use these operators to \emph{define} a set of $a_j$
operators acting on $C^{2^n}$, and satisfying the Fermionic CCRs.

To begin, suppose for the sake of argument that we have found such a
set of operators.  Then from our earlier discussion the action of the
$a_j$ operators in the occupation number representation
$|\alpha\rangle = |\alpha_1,\ldots,\alpha_n\rangle$ must be as
follows:
\begin{itemize}
\item Suppose $\alpha_j = 0$.  Then $a_j|\alpha\rangle = 0$.
  
\item Suppose $\alpha_j = 1$.  Let $\alpha'$ be that vector which
  results when the $j$th entry of $\alpha$ is changed to $0$.  Then
  $a_j|\alpha\rangle = -(-1)^{s_\alpha^j} |\alpha'\rangle$, where
  $s_\alpha^j \equiv \sum_{k=1}^{j-1} \alpha_k$.
\end{itemize}
If we identify the occupation number state $|\alpha\rangle$ with the
corresponding computational basis state $|\alpha\rangle$, then this
suggests taking as our \emph{definition}
\begin{eqnarray}
  a_j \equiv -\left( \otimes_{k=1}^{j-1} Z_k \right) \otimes \sigma_j,
\end{eqnarray}
where $\sigma_j$ is used to denote the matrix $\sigma \equiv |0\rangle
\langle 1|$ acting on the $j$th qubit.  It is easily verified that
these operators satisfy the Fermionic CCRs.  This definition of the
$a_j$ is known as the \emph{Jordan-Wigner transform}.  It allows us to
define a set of operators $a_j$ satisfying the Fermionic CCRs in terms
of the usual operators we use to describe qubits, or spin-$\frac 12$
systems.

The Jordan-Wigner transform can be inverted, allowing us to express
the Pauli operators in terms of the Fermionic operators
$a_1,\ldots,a_n$.  In particular, we have
\begin{eqnarray}
  Z_j = a_ja_j^\dagger-a_j^\dagger a_j.
\end{eqnarray}
This observation may also be used to obtain an expression for $X_j$ by
noting that $X_j = \sigma_j +\sigma_j^\dagger$, and thus:
\begin{eqnarray}
  X_j = -(Z_1 \ldots Z_{j-1}) (a_j+a_j^\dagger).
\end{eqnarray}
Substituting in the expressions for $Z_1,\ldots,Z_{j-1}$ in terms of
the Fermionic operators gives the desired expression for $X_j$ in
terms of the Fermionic operators.  Similarly, we have
\begin{eqnarray}
  Y_j = i (Z_1 \ldots Z_{j-1}) (a_{j}^\dagger-a_{j}),
\end{eqnarray}
which, together with the expression for the $Z_j$ operators, enables
us to express $Y_j$ solely in terms of the Fermionic operators.

These expressions for $X_j$ and $Y_j$ are rather inconvenient,
involving as they do products of large numbers of Fermi operators.
Remarkably, however, for certain simple \emph{products} of Pauli
operators it is possible to obtain quite simple expressions in terms
of the Fermi operators.  In particular, with a little algebra we see
that:
$$
  Z_j = a_ja_j^\dagger - a_j^\dagger a_j
$$
$$
  X_jX_{j+1} = (a_j^\dagger-a_j)(a_{j+1}+a_{j+1}^\dagger )
$$
$$
  Y_jY_{j+1} = -(a_j^\dagger+a_j)(a_{j+1}^\dagger-a_{j+1})
$$
$$
  X_jY_{j+1} =  i(a_j^\dagger-a_j) (a_{j+1}^\dagger-a_{j+1})
$$
$$
  Y_jX_{j+1} = i(a_j^\dagger+a_j) (a_{j+1}^\dagger+a_{j+1}).
$$

Suppose now that we have an $n$-qubit Hamiltonian $H$ that can be
expressed as a sum over operators from the set $Z_j,
X_jX_{j+1},Y_jY_{j+1},X_jY_{j+1}$ and $Y_{j}X_{j+1}$.  An example of
such a Hamiltonian is the transverse Ising model,
\begin{eqnarray}
  H = \alpha \sum_j Z_j + \beta \sum_j X_j X_{j+1},
\end{eqnarray}
which describes a system of magnetic spins with nearest neighbour
couplings of strength $\beta$ along their $\hat x$ axes, and in an
external magnetic field of strength $\alpha$ along the $\hat z$ axis.

For any such Hamiltonian, we see that it is possible to re-express the
Hamiltonian as a Fermi quadratic Hamiltonian.  As we saw in an earlier
section, determining the energy levels is then a simple matter of
finding the eigenvalues of a $2n \times 2n$ matrix, which can be done
very quickly. In particular, finding the ground state energy is simply
a matter of finding the smallest eigenvalue of that matrix, which is
often particularly easy.  In the case of models like the transverse
Ising model, it is even possible to do this diagonalization
analytically, giving rise to exact expressions for the energy
spectrum.  Details can be found in the paper by Lieb, Schulz and
Mattis mentioned earlier, or books such as the well-known book by
Sachdev on quantum phase transitions.

\textbf{Exercise:} What other products of Pauli operators can be
expressed as quadratics in Fermi operators?

\textbf{Problem:} I've made some pretty vague statements about finding
the spectrum of a matrix being ``easy''.  However, I must admit that
I'm speaking empirically here, in the sense that in practice I know
this is easily done on a computer, but I don't know a whole lot about
the computational complexity of the problem.  One obvious observation
is that finding the spectrum is equivalent to finding the roots of the
characteristic equation, which is easily computed, so the problem may
be viewed as being about the computational complexity of root-finding.

%#post_footer: <em>Finis</em>
%#end_post_footer

%#post_break

%#post_header: This is a little appendix to my post about the
%# consequences of the fermionic CCRs.  The results described in the
%# appendix are well-known, but I'm rather fond of the little proof given,
%# and so am indulging myself by including it here.
%#end_post_header

\section{Appendix on mutually commuting observables}

Any undergraduate quantum mechanics course covers the fact that a
mutually commuting set of Hermitian operators possesses a common
eigenbasis.  Unfortunately, in my experience this fact is usually
proved rather early on, and suffers from being presented in a slightly
\emph{too} elementary fashion, with inductive constructions of
explicit basis sets and so on.  The following proof is still
elementary, but from a slightly more sophisticated perspective.  It
is, I like to imagine, rather more like what would be given in an
advanced course in linear algebra, were linear algebraists to actually
cover this kind of material.  (They don't, so far as I know, having
other fish to fry.)

Suppose $H_1,\ldots,H_m$ are commuting Hermitian (indeed, normal
suffices) operators with spectral decompositions:
\begin{eqnarray}
  H_j = \sum_{jk} E_{jk} P_{jk},
\end{eqnarray}
where $E_{jk}$ are the eigenvalues of $H_j$, and $P_{jk}$ are the
corresponding projectors.  Since the $H_j$ commute, it is not
difficult to verify that for any quadruple $j,k,j',k'$ the operators
$P_{jk}$ and $P_{j'k'}$ also commute.  For a vector $\vec k =
(k_1,\ldots,k_m)$ define the operator
\begin{eqnarray}
  P_{\vec k} \equiv P_{1 k_1} P_{2 k_2} \ldots P_{m k_m}.
\end{eqnarray}
Note that the order of the operators on the right-hand side does not
matter, since they all commute with one another.  The following
equations all follow easily by direct computation, the mutual
commutativity of the $P_{jk}$ operators, and standard properties of
the spectral decomposition:
\begin{eqnarray}
  P_{\vec k}^\dagger = P_{\vec k}; \,\,\,\, \sum_{\vec k} P_{\vec k} = I; \,\,\,\, P_{\vec k} P_{\vec k'} = \delta_{\vec k \vec k'} P_{\vec k}.
\end{eqnarray}
Thus, the operators $P_{\vec k}$ form a complete set of orthonormal
projectors.  Furthermore, suppose we have $P_{\vec k} |\psi\rangle =
|\psi\rangle$.  Then we will show that for any $j$ we have $P_{jk_j}
|\psi\rangle = |\psi\rangle$, so $|\psi\rangle$ is an eigenstate of
$H_j$ with eigenvalue $k_j$.  This shows that the operators $P_{\vec
  k}$ project onto a complete orthonormal set of simultaneous
eigenspaces for the $H_j$, and will complete the proof.

Our goal is to show that if $P_{\vec k} |\psi\rangle = |\psi\rangle$
then for any $j$ we have $P_{jk_j} |\psi\rangle = |\psi\rangle$.  To
see this, simply multiply both sides of $P_{\vec k} |\psi\rangle =
|\psi\rangle$ by $P_{jk_j}$, and observe that $P_{jk_j} P_{\vec k} =
P_{\vec k}$.  This gives $P_{\vec k}|\psi\rangle =
P_{jk_j}|\psi\rangle$.  But $P_{\vec k}|\psi\rangle = |\psi\rangle$,
so we obtain $|\psi\rangle = P_{j k_j}|\psi\rangle$, which completes
the proof.

%\bibliography{../../mybib}

\end{document}